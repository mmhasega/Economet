% Options for packages loaded elsewhere
\PassOptionsToPackage{unicode}{hyperref}
\PassOptionsToPackage{hyphens}{url}
%
\documentclass[
]{book}
\usepackage{lmodern}
\usepackage{amssymb,amsmath}
\usepackage{ifxetex,ifluatex}
\ifnum 0\ifxetex 1\fi\ifluatex 1\fi=0 % if pdftex
  \usepackage[T1]{fontenc}
  \usepackage[utf8]{inputenc}
  \usepackage{textcomp} % provide euro and other symbols
\else % if luatex or xetex
  \usepackage{unicode-math}
  \defaultfontfeatures{Scale=MatchLowercase}
  \defaultfontfeatures[\rmfamily]{Ligatures=TeX,Scale=1}
\fi
% Use upquote if available, for straight quotes in verbatim environments
\IfFileExists{upquote.sty}{\usepackage{upquote}}{}
\IfFileExists{microtype.sty}{% use microtype if available
  \usepackage[]{microtype}
  \UseMicrotypeSet[protrusion]{basicmath} % disable protrusion for tt fonts
}{}
\makeatletter
\@ifundefined{KOMAClassName}{% if non-KOMA class
  \IfFileExists{parskip.sty}{%
    \usepackage{parskip}
  }{% else
    \setlength{\parindent}{0pt}
    \setlength{\parskip}{6pt plus 2pt minus 1pt}}
}{% if KOMA class
  \KOMAoptions{parskip=half}}
\makeatother
\usepackage{xcolor}
\IfFileExists{xurl.sty}{\usepackage{xurl}}{} % add URL line breaks if available
\IfFileExists{bookmark.sty}{\usepackage{bookmark}}{\usepackage{hyperref}}
\hypersetup{
  pdftitle={Notas de Aulas de Econometria},
  pdfauthor={Marcos Minoru Hasegawa},
  hidelinks,
  pdfcreator={LaTeX via pandoc}}
\urlstyle{same} % disable monospaced font for URLs
\usepackage{longtable,booktabs}
% Correct order of tables after \paragraph or \subparagraph
\usepackage{etoolbox}
\makeatletter
\patchcmd\longtable{\par}{\if@noskipsec\mbox{}\fi\par}{}{}
\makeatother
% Allow footnotes in longtable head/foot
\IfFileExists{footnotehyper.sty}{\usepackage{footnotehyper}}{\usepackage{footnote}}
\makesavenoteenv{longtable}
\usepackage{graphicx,grffile}
\makeatletter
\def\maxwidth{\ifdim\Gin@nat@width>\linewidth\linewidth\else\Gin@nat@width\fi}
\def\maxheight{\ifdim\Gin@nat@height>\textheight\textheight\else\Gin@nat@height\fi}
\makeatother
% Scale images if necessary, so that they will not overflow the page
% margins by default, and it is still possible to overwrite the defaults
% using explicit options in \includegraphics[width, height, ...]{}
\setkeys{Gin}{width=\maxwidth,height=\maxheight,keepaspectratio}
% Set default figure placement to htbp
\makeatletter
\def\fps@figure{htbp}
\makeatother
\setlength{\emergencystretch}{3em} % prevent overfull lines
\providecommand{\tightlist}{%
  \setlength{\itemsep}{0pt}\setlength{\parskip}{0pt}}
\setcounter{secnumdepth}{5}
\usepackage[brazil]{babel}
\usepackage[utf8]{inputenc}
\usepackage[T1]{fontenc}
\usepackage{lmodern}
\usepackage{amsmath,amssymb,amsthm,adjustbox,mathtools,bm,cool}
\usepackage{natbib}
\usepackage{graphics,graphicx,import,color,float}
\usepackage{url,booktabs,siunitx}
\usepackage[]{natbib}
\bibliographystyle{apalike}

\title{Notas de Aulas de Econometria}
\author{Marcos Minoru Hasegawa}
\date{2020-09-14}

\begin{document}
\maketitle

{
\setcounter{tocdepth}{1}
\tableofcontents
}
\hypertarget{licenuxe7a}{%
\chapter*{Licença}\label{licenuxe7a}}
\addcontentsline{toc}{chapter}{Licença}

Como está descrito no repositório, os poucos códigos originais desenvolvidos ao longo do texto estão sob a licença \textbf{GNU GPLv3} .

O texto e as artes gráficas elaboradas de forma original estão sob licença \textbf{Creative Commons BY-NC-SA 4.0}.

\hypertarget{sobre-o-material}{%
\chapter*{Sobre o material}\label{sobre-o-material}}
\addcontentsline{toc}{chapter}{Sobre o material}

A situação especial causada pela pandemia da COVID-19 forçou a muitos professores criarem materiais para facilitar aulas remotas das suas disciplinas. A disciplina SE308 Econometria da UFPR não poderia ser diferente. Então, o objetivo deste material é de suprir a falta das bibliografias básicas na sua versão digital com a disponibilização de forma digital e gratuita o que seria o material das notas das aulas da disciplina de Econometria. Não é o ideal, mas a ideia é melhorar o material com tempo.

\hypertarget{sobre-o-autor}{%
\chapter*{Sobre o Autor}\label{sobre-o-autor}}
\addcontentsline{toc}{chapter}{Sobre o Autor}

Professor do Departamento de Economia da Universidade Federal do Paraná. Engenheiro Agrônomo pela UNESP/Jaboticabal, Mestrado em Economia Agrária pela ESALQ/USP e Doutorado em Economia Aplicada pela ESALQ/USP, é um dos professores responsáveis pelas disciplinas de SE305 Estatística Econômica e Introdução à Econometria e SE308 Econometria ambas do curso de Economia da Universidade Federal do Paraná (UFPR).

\hypertarget{amostrafinita}{%
\chapter{Propriedades de amostra finita do MQO}\label{amostrafinita}}

No capitulo \ref{amostrafinita} apresena as prorpriedades de pequena amostra ou amostra finita do estimador de mínimos quadrados ordinários. Como a maior parte deste material, o primeiro o capítulo \ref{amostrafinita} tem como base \citet{Hayashi2000}.

\hypertarget{o-modelo-regressuxe3o-linear-cluxe1ssico}{%
\section{O modelo regressão linear clássico}\label{o-modelo-regressuxe3o-linear-cluxe1ssico}}

O modelo de regeressão linear clássico, as variáveis, chamadas de variável dependente ou regressanda é relacionada com outras várias variáveis denominadas regressoras ou variáveis explicativas. Suponha que se observe \(n\) valores para estas variáveis. Seja \(y_i\) a \(i\)-ésima observação da variávei dependente em questãoe seja \((x_{i1},x_{i2},x_{i3}, \ldots, x_{iK})\) as \(i\)-ésimas observações dos \(K\) regressores. A \textbf{amostra} ou \textbf{dado} é a um coleção destas \(n\) observações.

\hypertarget{literature}{%
\chapter{Literature}\label{literature}}

Here is a review of existing methods.

\hypertarget{methods}{%
\chapter{Methods}\label{methods}}

We describe our methods in this chapter.

\hypertarget{applications}{%
\chapter{Applications}\label{applications}}

Some \emph{significant} applications are demonstrated in this chapter.

\hypertarget{example-one}{%
\section{Example one}\label{example-one}}

\hypertarget{example-two}{%
\section{Example two}\label{example-two}}

\hypertarget{final-words}{%
\chapter{Final Words}\label{final-words}}

We have finished a nice book.

  \bibliography{economet.bib,book.bib,packages.bib}

\end{document}
